\subsection{Google SketchUp}
\subsubsection{Allgemein}
Google SketchUp ist eine von der Firma Google entwickelte Anwendung zur Erstellung von dreidimensionalen Modellen am Computer. Durch das zur Verfügung stellen einfach zu bedienender Tools und eine sehr logische Bedienweise ist das Programm verglichen mit anderen Tools zur 3D Modellierung auch für Anfänger gut geeignet. Trotzdem lassen sich bei richtiger Bedienung präzise technische Zeichnungen anfertigen.

Das Hauptaugenmerk von Google SketchUp liegt allerdings darauf, die schnelle Modellierung von Gebäuden und sonstiger Architektur zu ermöglichen und es möglichst einfach zu machen das Modell mit Texturen zu versehen. Mit der Bereitstellung dieses Tools hofft Google, Nutzer anregen zu können Gebäude in ihrer Umgebung zu modellieren und dem Google Dienst Google Earth hinzuzufügen. \footcite[vgl.][]{sketchupchip} \footcite[vgl.][]{sketchupwiki}

\subsubsection{Einsatz im Projekt}
In unserem Projekt kam Google SketchUp ausschließlich zur Erstellung eines schematischen Robotermodells für die Visualisierung, sowie zur Konstruktion des Edubot-Modells zum Einsatz. Auch dieses Modell ist separat als Visualisierung verfügbar. 
Es ist mit dieser Software möglich, die erstellten Modelle im "'.obj"' Format zu speichern, welches Problemlos durch Microsoft Expression Blend geöffnet und in XAML konvertiert werden kann. 

\subsubsection{Auswahlkriterien und Alternativen}
Der Hauptgrund für die Wahl von Google SketchUp als Modellierungsprogramm war die Einfachheit der Benutzung. Darüber hinaus verfügten wir bereits über Erfahrungen in der Arbeit mit dem Programm.

Als Alternative wäre vor allem die freie Software "'Blender"' in Frage gekommen. Dieser Gedanke wurde allerdings im Hinblick auf die Komplexität der Bedienung schnell verworfen.
 