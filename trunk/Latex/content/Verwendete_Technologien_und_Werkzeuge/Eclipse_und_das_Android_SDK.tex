
\subsection{Eclipse und das Android SDK}

\subsubsection{Eclipse IDE}
Die Eclipse IDE ist eine Open Source Entwicklungsumgebung zur Entwicklung von Software verschiedenster Art. Sie ist die meist verwendete Entwicklungsumgebung für die Programmiersprache Java. Es gibt zahlreiche Plugins, mit denen sie um Fähigkeiten erweitert werden kann. Seit dem Erscheinen von Android ist sie auch die offizielle Entwicklungsumgebung für Androidprogramme. \footcite[vgl.][]{eclipse} 

\begin{figure}[htbp]
\centering
  \fbox{
    \includegraphics[width=12cm]{images/eclipse}
  }
\caption{Eclipse IDE}
\end{figure}

\subsubsection{Android SDK und Plugin}
Um eigene Programme entwickeln zu können, benötigt man zusätzlich zum Java-SDK auch das Android-SDK, dies befindet sich in den Android Development Tools,kurz ADT, diese wiederum stehen auf der Android Developer Homepage zum Download bereit. Es beinhaltet den Cross-Assembler der den von Java-Compiler übersetzten Quellcode für Dalvik-VM verständlich macht.\footcite[vgl.][]{androidsdk}

Weiters beinhaltet das Android SDK:

\begin{itemize}

\item Android Virtual Device Manager \\
Mit diesem Tool können die einzelnen virtuellen Geräte verwaltet und konfiguriert werden. Diese virtuellen Geräte können dann mit dem Android Emulator gestartet werden.

\item Dalvik Debug Monitor Server \\
Mit diesem Tool ist es möglich die entwickelten Applikationen zu Debuggen um Fehler im Programm zu finden. Er beinhaltet unter anderem auch die Logcat, die das Auslesen der System-Logs ermöglich. Weiters kann auch direkt, falls freigegeben, auf das File-System zugegriffen werden, um Daten zu betrachten oder um sie zu verändern.


\item Android Emulator \\
Der Emulator ist ein Programm mit dem es möglich ist die Android Virtual Devices am Computer auszuführen. Dieser kann in Größe und mit einzelnen Features angepasst werden.


\item Viele andere kleine Tools, die für das Entwickeln von Android-Applikationen von Nutzen sein können.
\end{itemize}

