
\subsection{SCRUM}
Scrum ist ein agiles Vorgehensmodel zur Entwicklung von Software, das mit wenigen klaren Regeln auskommt. Als agiles Vorgehensmodel verkörpert Scrum die Werte des Agilen Manifests.
Dieses Manifest stellt die Menschen in den Mittelpunkt der Softwareentwicklung, da durch ihre Kooperation und durch ihr Engagement die Software erst entsteht. Außerdem stellt das Manifest die Kundenzufriedenheit in den Vordergrund, den dieser ist es, der später das Produkt kauft und damit zufrieden sein möchte. \footcite[vgl.][]{scrum}

Wie schon erwähnt besteht SCRUM aus wenigen klaren Regeln, diese jedoch müssen klar eingehalten werden, ansonsten funktioniert das gesamte System nicht. Ein wesentlicher Punkt dabei ist die klare Definition der einzelnen Rollen.

\subparagraph{Rollen in Scrum\footcite[vgl.][]{scrumroles}}
\begin{itemize}
\item{\textbf{Der Product Owner}}\\
Der Product Owner repräsentiert in einem Projekt den Endkunden. Er steuert die Softwareentwicklung und arbeitet während des gesamten Projektverlaufes mit dem Team eng zusammen. Er beeinflusst von Anfang an den Erfolg des Projektes und ist auch für das gesamte Projekt verantwortlich. Er muss die Bedürfnisse der Kunden erfassen und diese an das Team als Anforderungen weitergeben.
\item{\textbf{Das Team}}\\
Das Team ist repräsentiert die Entwickler in dem Projekt und es ist somit für die Umsetzung des Projektes zuständig. Die Entwickler arbeiten dabei sehr eng zusammen und jedes Mitglied sollte zu jedem Zeitpunkt wissen, was der Andere gerade macht. Das Team ist eigenverantwortlich und entscheidet selbst welche Anforderungen in einem Sprint bewältigt werden können. 
\item{\textbf{Der ScrumMaster}}\\
Der ScrumMaster ist der Trainer der gesamten SCRUM-Mannschaft. Er hilft dem Team dabei Scrum richtig einzusetzen. Er ist dafür zuständig das alle Scrum Regeln genau eingehalten werden. Sein Ziel sollte sein, sich im Laufe des Projekt überflüssig zu machen, da jeder von sich aus die Regeln automatisch einhalten sollte.
 
\end{itemize}

\subparagraph{Der Ablauf eines Scrum-Projektes \footcite[vgl.][]{scrumprocess}}
Um den erfolgreichen Ausgang eines Scrum-Projektes zu gewährleisten, muss der vorgegebene Ablauf genauestens eingehalten werden.
\begin{itemize}
\item{\textbf{Erstellung des Product Backlogs}}\\
Der Product Backlog ist eine Sammlung aller Anforderungen. Diese werden entweder in Form von Karteikarten oder elektronisch als Tabelle dargestellt. Um den Product Backlog zu füllen wird zu aller erst ein Produktkonzept entwickelt. Dieses Produktkonzept beinhaltet die Idee des Programms mit einer Berücksichtigung auf die Bedürfnisse der Kunden. Dieses Produktkonzept sollte dann für den ProductBacklog übernommen werden. Wichtig ist dabei, dass dieser übersichtlich bleibt, da man sonst leicht die Orientierung verliert.


\item{\textbf{Die Sprint Planung}}\\
Nach der Erstellung des Product Backlogs wird dieser in wichtige Teilbereiche, sogenannte User Stories aufgeteilt. Jede solche User Story sollte bei Abschluss, einen Mehrwert für das Produkt und dem Kunden bietet. Diese User Stories werden dann auf die einzelnen Sprints aufgeteilt. Dabei wird vor jedem Sprint vom Team geschätzt, wie viele User Stories man in einem Sprint schaffen kann. Mangels Erfahrung ist dies bei den ersten Sprints noch relativ schwierig, jedoch im Laufe des Projektes, schätzt sich das Team immer besser ein und der ProductOwner weiß ziemlich genau, was sein Team die nächsten Wochen schaffen wird.
\item{\textbf{Der Sprint}}\\
Die Umsetzung des Programms findet im sogenannten Sprint statt. Ein Sprint dauert immer gleich viele Tage und sollte maximal 30 Tage dauern. In unserem Fall dauerte ein Sprint genau 14 Tage, also 10 Arbeitstage, dies ist auch der Durchschnittswert der bei anderen Projekten gewählt wird. In diesem Zeitraum werden die User Stories vom Team in Task aufgeteilt. Jedes Teammitglied kann zwischen den Tasks wählen und diese erledigen. Nach jedem Sprint gibt es ein Review-Meeting, wo die erledigten Stories dem ProductOwner präsentiert werden. In diesem Review-Meeting sollte immer eine auslieferbare Version des Programms vorzeigbar sein, daraus ist zu schließen, dass es keine Fehler beinhalten soll. Daher sollte sich das Team immer einen halben Tag vor dem Review-Meeting Zeit nehmen, um das Produkt ausgiebig zu Testen. 
Nach jedem Sprint sollte außerdem eine Sprint-Retrosprektive durchgeführt werden. In dieser Retrosprektive hat das Team die Möglichkeit, Verbesserungsvorschläge für den Entwicklungsprozess mit einzubringen und über diese Vorschläge dann untereinander zu diskutieren. 



\item{\textbf{Daily Scrum}}\\
Die Daily-Scrum-Besprechung ist eine Besprechung am Anfang jedes Arbeitstages. In dieser Besprechung erzählt jedes Teammitglied an was er gerade arbeitet. Dies hat den Sinn, dass zu jedem Zeitpunkt, jedes Teammitglied weis, an was der Andere gerade arbeitet. Somit könnte zu jeder Zeit, ein Teammitglied, für ein anderes Mitglied einspringen.


\end{itemize}

\begin{figure}[H]
\centering
\includegraphics[width=15cm]{images/scrum}
\caption{Der Ablauf eines Scrum-Projektes}
\end{figure}
\newpage


