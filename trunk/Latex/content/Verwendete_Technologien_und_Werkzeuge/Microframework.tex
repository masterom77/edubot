\subsection{.Net Micro Framework}
%http://www.microsoft.com/en-us/netmf/default.aspx
%http://de.wikipedia.org/wiki/.NET_Micro_Framework
\subsubsection{Allgemein}
Das .Net Micro Framework wurde als Abkömmling des .Net Framewoks von Microsoft entwickelt und für die Entwicklung von Anwendungen auf kleinen Geräten mit geringen Ressourcen Konzipiert. Damit eignet es sich auch für die Entwicklung von Software für Mikrocontroller. Grundsätzlich wird mit dem .Net Micro Framework versucht das Programmieren von Mikrocontrollern einfacher und komfortabler zu gestallten. Derzeit befindet sich das .Net Micro Framework in Version 4.2 und wird nach Installation der entsprechenden SDK problemlos von Visual Studio 2010 unterstützt. 
Zusätzlich zur SDK ist zur Entwicklung eine Programmbibliothek des jeweiligen Hardwareherstellers nötig um auf die spezifischen Funktionen des Controllers Zugriff zu erhalten.
Das .Net Micro Framework ist in seinem Grundaufbau nicht Echtzeitfähig, zeitkritische Operation können jedoch über entsprechende Funktionen der Hersteller ausgegliedert und als nativer Code auf dem Controller ausgeführt werden um ausreichende Performance sicherzustellen. 
\subsubsection{Einsatz im Projekt}
Da es sich bei dem von uns beim Edubot Modell verwendeten Mikrocontroller um ein das .Net Micro Framework unterstütztes Modell handelt, konnte dieser mithilfe dieses Frameworks programmiert werden. 
\subsubsection{Alternativen und Auswahlkriterien}
Für die Programmierung eines Mikrocontrollers gibt es zahlreiche Alternativen, wie etwa die direkte Programmierung in C oder einer anderen Hardwarenahen Sprache. Da sich unser Wissen bezüglich dieser Hardwarenahen Programmierung auf das Basiswissen beschränkt und außerdem bei keiner der vorhandenen Alternativen das Vorgehen so Komfortabel ist, wählten wir das .Net Micro Framework für die Aufgabe der Mikrocontroller Programmierung aus. 
Ein weiterer Grund für die Auswahl war, dass der Zugang zu einem passenden Controller sehr schnell Möglich war und bereits einige Tests mit derartiger Hardware durchgeführt worden waren.
