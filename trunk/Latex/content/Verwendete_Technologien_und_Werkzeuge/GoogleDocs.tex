\subsection{GoogleDocs}
\subsubsection{Allgemein}
Unter dem Namen Google Docs stellt der Internetdienstleister Google ein Webbasiertes Office Paket zur Verfügung. Google Docs unterstützt zahlreiche Dateiformate und bietet eigene Formate die für die Onlineverarbeitung optimiert sind an. Jeder angemeldete Nutzer hat für die Verwendung des Office Pakets etwa 5 GB Speicherplatz zur Verfügung und kann alle seine Dokumente online Speichern. 
Gespeicherte Dokumente können anderen Nutzern zugänglich gemacht werden und in sogenannten Sammlungen die denselben Zweck erfüllen wie Ordner organisiert werden.
Für die Bearbeitung der Dokumente stehen verschiedene Editoren zur Verfügung, deren Aufzählung hier nicht sinnvoll wäre, da sich sowohl ihre Bezeichnungen und ihre Anzahl stetig verändert.
Eine sehr wichtige Funktion von Google Docs ist die in einigen Editoren, so auch im Texteditor, vorhandene Möglichkeit Dokumente gleichzeitig zu bearbeiten und in Echtzeit alle durch andere Benutzer getätigte Änderungen zu verfolgen.
\subsubsection{Einsatz im Projekt}
Google Docs kam im Verlauf dieses Projekts, mit Ausnahme dieser schriftlichen Ausarbeitung, überall dort zum Einsatz, wo Dokumente anzufertigen waren, Grafiken erstellt wurden oder Ideensammlungen entstanden. Durch die Möglichkeit gleichzeitig und in Echtzeit am selben Dokument zu arbeiten war eine sehr produktive Arbeitsweise möglich und vor allem die Zusammenarbeit wurde erleichtert.  
\subsubsection{Alternativen}
Im Internet gibt es zahlreiche Alternativen zum Angebot von Google, zu erwähnen sind an dieser Stelle vor allem die Anwendung  „‘Buzzworld“‘ von Adobe, OpenGoo und der Webdienst Zoho. Im Vorfeld dieses Projekts wurde jedoch keine dieser Alternativen näher in Betracht gezogen, da wir in der Vergangenheit bei der Verwendung von Google Docs sehr gute Erfahrungen machen konnten und zusätzlich unsere gesamte Dokumentenverwaltung auch außerhalb dieses Projekts auf Google Docs basiert. Ein Zusätzliches Werkzeug hätte die Verwaltung aller persönlichen Dokumente wesentlich unübersichtlicher gemacht.
