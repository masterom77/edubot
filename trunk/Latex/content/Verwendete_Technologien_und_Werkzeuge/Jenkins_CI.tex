
\subsection{Jenkins CI}
Jenkins ist ein erweiterbares, webbasiertes System welches von Kohsuke Kawaguchi, einem Mitarbeiter von Sun Microsystems entwickelt wurde. Mit Jenkins CI ist es möglich den  Prozess der kontinuierlichen Integration zu automatisieren.
Unsere Applikation wurde mittels Jenkins gebaut und mittels von uns implementierten Testrunner automatisch getestet.\footcite[vgl.][]{jenkins}

\subparagraph{Die kontinuierliche Integration}

Unter kontinuierlicher Integration versteht man das regelmäßige und vollständige Neubilden und Testen der Applikation.\footcite[vgl.][]{ci}
 
\textbf{Das Konzept:}
\begin{itemize}
\item Jeder Entwickler soll oft Änderungen in die Versionsverwaltung einchecken d.h. er sollte versuchen, jeden fertigen kleinen Teil einzuchecken. 
\item Sobald ein Entwickler Änderungen in die Versionsverwaltung eincheckt, wird die Applikation gebaut und es startet automatisch die geschriebenen Tests.
\end{itemize}

\begin{figure}[htbp]
  \centering
  \begin{minipage}[t]{7 cm}
    \fbox{
    	\includegraphics[width=7cm]{images/jenkins_dashboard} 
    } 
    \caption{Jenkins: Hauptseite}
  \end{minipage}
  \hspace{0.5cm}
  \begin{minipage}[t]{7 cm}
  	\fbox{
	    \includegraphics[width=7cm]{images/jenkins_monitor}  
    }
    \caption{Jenkins: Test und Build Status}
  \end{minipage}
  
\end{figure}