
\subsection{Subversion}

\subsubsection{Allgemein}
Subversion ist ein Tool zur Versionsverwaltung von Dateien und Verzeichnissen. Es wird von der Firma CollabNet vertrieben und steht unter der Apache Open Source License. Subversion ermöglicht, dass Entwickler am selben Code arbeiten können.\footcite[vgl.][]{subversion} 

\subparagraph{Die Grundbefehle von Subversion:}
\begin{itemize}
\shorthandoff{"}
\item SVN-Checkout:\\
Dies ist der erste Schritt, um eine lokale Kopie der Daten zu erstellen. Hierbei werden alle Daten vom Server(Repository) heruntergeladen und es entsteht eine "Working Copy" auf dem lokalen Rechner. Nun kann der Entwickler mit den lokalen Daten arbeiten und diese verändern.

\item SVN-Commit:\\
Wenn der Entwickler es möchte, dass seine Änderungen auch am Server geändert werden, macht er einen Commit. Hierbei werden die veränderten Daten mit dem Server abgeglichen. Dazu muss aber die lokale Kopie auf dem selben Stand sein, wie die auf dem Server. Ist dies nicht der Fall muss ein SVN-Update durchgeführt werden.

\item SVN-Update:\\
Mit diesem Befehl werden die lokalen Daten mit denen auf dem Server abgeglichen. Hierbei kann es zu Konflikten mit den lokal bearbeiteten Daten kommen. Diese müssen dann von dem Entwickler behoben werden. 
\end{itemize}

Subversion kann über die Konsole oder mit entsprechenden Tools verwendet werden.

\subsubsection{TortoiseSVN}
TortoiseSVN ist ein Subversion Client für Windows. Dieser Client ist nahtlos in den Windows Explorer eingebunden. Er ist somit immer mittels Kontextmenü aufrufbar und erkennt sofort, ob es sich bei der ausgewählten Datei, um eine Datei eines SVN-Repository handelt. Wir haben diesen Client auf den PCs der Firma verwendet, da wir aber nach den Ferien auf unsere Macbooks umgestiegen sind, wurde dieser Client später nicht mehr verwendet.\footcite[vgl.][]{tortoisesvn}


\subsubsection{Versions}

Versions ist ein sehr einfach bedienbarer Subversion Client für den Mac. Die Benutzeroberfläche ist selbsterklärend und der Client integriert sich perfekt in das Betriebssystem. Da dieser jedoch ziemlich langsam ist und eine Lizenz 59\$ kosten würde, verwendeten wir den Client nur bis zum Ende des Probezeitraumes und gingen wieder zurück zur Console.\footcite[vgl.][]{versions}




\subsubsection{Subclipse}
Subclipse ist ein Plugin für Eclipse, welches den Einsatz von Subversion in Eclipse möglich macht. Es ermöglicht das SVN-Aktionen wie Commit, Update, Revert und Cleanup direkt auf das jeweilige Eclipseprojekt angewendet werden können. Zusätzlich unterstützt es den Entwickler mit den Zusatzfunktionen, wie dem Revision Graphen und der Historyansicht.\footcite[vgl.][]{subclipse}


