
\subsection{Event-System}

\subsubsection{Aufgaben}
Es soll möglich sein über bestimmte Ereignisse während des Betriebs auch softwaretechnisch reagieren zu. Dadurch wird die API um ein weiteres Anwendungspektrum erweitert und ermöglicht beispielsweise eine vollständige Überwachung der Zustände durch eine, die API-verwendende, Drittsoftware.

\subsubsection{Aufbau}
Um diese Aufgabe erfüllen zu können, wird das Observer-Pattern verwendet, dessen Umsetzung in C\# durch sogenannte Delegates (siehe Listing) erleichtert wird. In der API wird dazu das Event-Delegate verwendet, welches die Struktur einer Methode, welche bei Eintreten eines Ereignisses ausgeführt werden sollen, definiert. 
\newline
Die abstrakte Klasse IAdapter besitzt für jedes wichtige Ereignis ein, mit dem Schlüsselwort \textbf{event} gekennzeichnetes, Property vom Delegate Event, welchem benutzerdefinierten Methoden, die Eintreten des Events ausgeführt werden sollen, mitgegeben werden können. Somit kann ein Entwickler nach seinen Vorstellungen auf bestimmte Ereignisse eines Adapters reagieren.

\subsubsection{Umsetzung}
\textbf{Event}
\newline
Das Event-Delegate definiert folgende Methoden-Struktur:

\begin{lstlisting}[language = CSharp, captionpos=b, caption={Das Event-Delegate}]
public delegate void Event(object src, System.EventArgs args);
\end{lstlisting}

Die Methode besitzt keinen Rückgabewert (void) und übernimmt als ersten Parameter ein Object sender, welches den Auslöser des Events enthält. Als zweiten Parameter muss die Methode ein EventArgs-Object args, welches zusätzliche Informationen zum Ereignis enthält, übernehmen.\\
\\
Die vorhandenen Events sowie der Zeitpunkt an dem diese ausgelöst werden sind in den Kapiteln Adapter-System und Command-System bereits detailliert beschrieben, werden jedoch aus Gründen der Übersichtlichkeit jedoch in der nachstehenden Tabelle nochmals zusammengefasst.

\begin{table}
\begin{tabular}{|p{5cm}|p{9cm}|}
\hline \rowcolor{lightgray}
\textbf{Ereignis} & \textbf{Zeitpunkt des Eintretens}\\
\hline
\textit{OnHoming} & Bei Ausführung eines InitCommands, nach erfolgreicher Zustandsvalidierung.\\
\hline
\textit{OnMovementStarted} & Bei Ausführung eines MVS- oder MVCCommands, nach erfolgreicher Zustandsvalidierung und Pfadberechnung.\\
\hline
\textit{OnToolUsed} & Bei Ausführung eines UseToolCommands, nach erfolgreicher Zustandsvalidierung.\\
\hline
\textit{OnShuttingDown} & Bei Ausführung eines ShutdownCommands, nach erfolgreicher Zustandsvalidierung.\\
\hline
\textit{OnAbort} & Bei Ausführung eines AbortCommands, nach Leeren der Warteschlange\\ 
\hline
\textit{OnStateChanged} & Bei jeder Änderung des Zustands-Properties des Adapters.\\
\hline
\textit{OnFailure} & Bei Auftreten eines Fehlers im jeweiligen Adapter\\
\hline
\end{tabular}
\caption{Überblick über mögliche Events}
\end{table}

\textbf{Event-Argumente}\\
Bei Auslösen eines Events werden im args-Parameter Details zum Event übergeben. Dazu werden verschiedene, von der Systemklasse EventArgs, abgeleitete Klassen verwendet, die im folgenden vorgestellt werden.
\begin{itemize}
\item \textbf{HomingEventArgs}
Eine Instanz dieser Klasse wird beim Auslösen des OnHoming-Events mitgegeben und enthält ein Property \textbf{CorrectionAngle} um den die Winkelstellungen bei jedem Schritt korrigiert werden soll. Im Falle des EdubotAdapters, wäre dieser beispielsweise 0,1125$^\circ$, was der Winkeländerung bei Ausführen eines Microschritts durch den Schrittmotor entspricht.
\item \textbf{FailureEventArgs}
Eine Instanz dieser Klasse wird beim Auslösen des OnFailure-Events als Parameter mitgegeben und dient zur Behandlung von Laufzeitfehlern. Das Property \textbf{ThrownException} enthält ein Exception-Objekt mit detaillierten Informationen zum jeweiligen Fehler. \\
\item \textbf{MovementStartedEventArgs}
Eine Instanz dieser Klasse wird beim Auslösen des OnMovementStarted-Events als Parameter mitgegeben und enthält ein InterpolationResult-Objekt mit den, für die Bewegung benötigten, Winkelstellungen. Es kann vorkommen das dieses Objekt null ist, da nur für Adapter deren \textbf{UsesIntegratedPathCalculation}-Methode \textbf{true} zurückgibt ein entsprechendes Objekt erzeugt wird. \\
\item \textbf{StateChangedEventArgs}
Eine Instanz dieser Klasse wird beim Auslösen des OnStateChanged-Events als Parameter mitgegeben und enthält den alten sowie den neue Zustand des Adapters in Form eines State-Objekts. Somit kann die genaue Zustandveränderung leicht nachvollzogen werden.\\
\item \textbf{ToolUsedEventArgs}
Eine Instanz dieser Klasse wird beim Auslösen des OnToolUsed-Events als Parameter mitgegeben und enthält einen Wert vom Typ Boolean der angibt ob das Werkzeug aktiviert oder deaktivert wurde.\\
\end{itemize}
