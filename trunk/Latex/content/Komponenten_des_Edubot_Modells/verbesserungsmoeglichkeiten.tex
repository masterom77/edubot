\subsection{Verbesserungsmöglichkeiten}
Beim Bau der Mechanik wurde sehr viel Wert auf Stabilität gelegt und die Gelenke, sowie die Achsen können wesentlich höheren Belastungen als des Haltens eines Stiftes standhalten. 
Dies wurde vor allem im Hinblick auf zukünftige Erweiterungen, die möglicherweise im Rahmen des Unterrichts entwickelt werden können, umgesetzt. Folgende Verbesserungen wären mit der Beschriebenen 

\begin{itemize}
\item \textbf{Übersetzung der Motoren}\\
Da die verwendeten Schrittmotoren bei entsprechender Ansteuerung eine wesentlich höhere Drehzahl ermöglichen würden, wäre es sinnvoll die Achsen nicht direkt, sondern über eine kleine Übersetzung zu befestigen. Dieser Schritt würde sich vor allem auf die Laufruhe (keine "eckigen" Bewegungen mehr), als auch auf die Kraft des Roboters positiv auswirken.

\item \textbf{Montage einer transversalen Achse}\\
Die Stabilität der aktuellen Konstruktion erlaubte es problemlos, eine dritte, transversale Achse am Ende der zweiten Achse anzubringen. Durch diesen Schritt wäre es möglich den Roboter einen Punkt im dreidimensionalen Raum anfahren zu lassen.
Als mögliche Ausgangsbasis für diesen Zusatz könnte die eine Konstruktion dienen wie sie in DVD und CD Laufwerken zur Positionierung des Lasers zu finden ist.
\end{itemize}