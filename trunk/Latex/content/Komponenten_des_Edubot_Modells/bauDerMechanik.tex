\subsection{Bau der Mechanik}
Beim Bau der Mechanik wurde großteils nach dem zuvor im Zuge der Konstruktionsplanung erstellten 3D Modell vorgegangen. Das Material für die Achsen wurde beinahe Ausschließlich aus 2,5 cm Dicken Dreischichtplatten gewonnen, als Wellenverlängerungen für die Motoren dienen Alurohre mit einem Durchmesser von 10 mm. 
Beim Bau des Roboters wurde in folgendenden Schritten vorgegangen:
\begin{itemize}
\item \textbf{Zurechtschneiden einer Grundplatte und Anbringung der Basis}\\
Der Erste Schritt beim Bau des Modells war das grobe Zurechtschneiden der Grundplatte, welche später die gesamte Arbeitsfläche des Armes Abdeken sollte und die Stabilität des Arms gewährleistet.
Direkt auf diese Grundplatte wurde dann die Basis des Roboters gschraubt. Bei der Basis handelt es sich um eine senkrecht aufgestellte Platte, an der später die Halterung für die primäre Achse befestigt wurde.
[Grafik]
\item \textbf{Bau des ersten Gelenks}\\
Das erste Gelenk des  Edubot Modells wurde direkt mit der Basis verschraubt und bietet damit einen sehr hohen Grad an Stabilität. Grundsätzlich basiert das Gelenk auf der Form eines liegenden U's welches über zwei Kugellager eine etwa 1 cm dicke Motorwellenverlängerung aus Alluminium hält. An diese Motorwellenverlängerung ist in weiterer Folge die primäre Achse des Motors so festgekelmmt, dass sie im 90$^\circ$ Winkel zur senkrecht stehenden Wellenverlängerung stehen.

\item \textbf{Befestigung der ersten Achse}\\
Für das Anklemmen der Achse an die Wellenverlängerung wurde wurde die Achse mit eiener 1 cm dicken dicken Bohrung versehen durch welche die Wellenverlängerung exakt passt. Zusätzlich wurde die Achse der länge durch die Mitte des Lochs etwa 4 cm eingeschnitten. Durch das spätere anbringen von Schrauben (entsprechende Bohrungen waren zuvor nötig), welche den entstandenen Spalt zusammenziehen können, wurde es möglich das gebohrte loch nach durchstecken der Wellenverlängerung zusätzlich zu verleinern und somit die Achse fest einzuklemmen.
Die erste Achse (primäre Achse) hat eine Länge von etwa 20 cm.
\item \textbf{Bau des Zweiten Gelenks}\\
Das zweite Gelenk wurde an direkt mit der primären Achse verschraubt und basiert wieder auf dem selben Prinzip wie das erste Gelenk. Beim Bau des zweiten Gelenks ergaben sich zwei kleine Unterschiede gegenüber dem großen Gelenk. Der erste Unterschied ist, dass das zweite Gelenk kleiner ist und aus Gewichtsgründen der Fokus nicht nur auf der Stabilität de Konstruktion lag. 
Der zweite Unterschied ist die Positionierung der Achse. Beim ersten Gelenk wurde die Achse zwischen den beiden Kugellagern, in der Mitte der U-Förmögen Konstruktion angebracht, beim zweiten Gelenk wurde die Achse unterhalb des U's angebracht. Diese Abänderung der ursprünglichen Konstruktion führt nur zu einem geringen stabiliätsverlust und erweitert den Arbeitsberreich des Roboters stark.
\item \textbf{Befestigung der zweiten Achse}\\
Bei der Befestigung der zweiten Achse (sekundär Achse) wurde im Prinzip gleich vorgegangen wie bei der ersten Achse (primär Achse), für eine genauere Beschreibung siehe den Punkt "Befestigung der ersten Achse" weiter oben in dieser Aufzählung.
\item \textbf{Befestigung der Motoren}\\
Der Motor der Primären Achse wurde direkt mit der Grundlatte des Motors verschraubt und läuft damit wenig Gefahr durch Verwindung oder Verrutschen die Positionierung des Arms zu verfälschen.
Der Motor der Sekundären Achse Achse wurde an das zweite Gelenk geschraubt, welches seinerseits direkt mit der primären Achse verbunden ist. Durch die Positionierung des zweiten Motors ohne zusätzlichen Abstsand direkt über dem zweiten Gelenk war es möglich die Verbindung mit der Wellenverlängerung direkt im Gelenk herzustellen und damit auf eine Komplexere Konstruktion zu verzichten. Die folgende Grafik zeigt am Beispiel des ersten Motors, wie die Verbindung der Motorwelle mit der Wellenverlängerung gelöst wurde: [todo]
\end{itemize}
\subsubsection{Planung und Bau der Elektronik}
Die Planung der Elektronik war ein stetiger Prozess und ging einher mit dem Versuchsweisen Aufbau der jeweiligen Planungsergebnisse. Grob unterteilen lässt sich die Entwicklung jedoch in zwei Phasen:
\begin{itemize}
\item \textbf{Versuchsweiser Aufbau mit direkter Ansteuerung}\\
In einem ersten Anlauf wurde der Versuch unternommen, die gesammte Ansteuerung der Schrittmotorsteuerungen und damit auch deren Versorgung mit dem Entsprechenden Taktsignal welches zum Verfahren der Schritte benötigt wird, über USB Modul für den PC zu realisieren. Der Vorteil dieser Konstruktion wäre gewesen, dass eine solche Ansteuerung wesentlich leichter zu realisieren gewesen wäre. Nach einigen Versuchen mit diesem Aufbau stellte sich jedoch heraus, dass durch mangelnde Echtzeitqualitäten von Microsoft Windows die Ausgabe des sehr hochfrequenten Taktsignals nicht stabil möglich war.
\item \textbf{Versuchsweiser Aufbau mit Mikrocontroller Steuerung}\\
Auf Basis der gewonnenen Erfahrungen aus dem Aufbau ohne Mikrocontroller, wurde nun überlegt wie das Taktsignal stabilisiert werden könnte und wie eine Ausgabe mit höherer Frequenz möglich werden würde. Es wurde schließlich beschlossen, die Ausgabe des Taktsignals, sowie sämtliche direkt mit der Elektronik des Roboters zusammenhängende Aufgaben von einem Mikrocontroller übernehmen zu lassen. Der Mikrocontroller wird über die Netzwerkschnittstelle mit Daten über die Verfahrbewegung versorgt und kann schließlich die Elektronik des Roboters autonom steuern.
\item \textbf{Endaufbau am Modell mit Endschaltern}\\
Nach Fertigstellung des Holzmodells wurde der zuerst nur Versuchsweise Aufbau der oben Beschriebenen Variante mit Mikrocontroller in dieses Modell eingebaut. In diesem Schritt wurden nun auch die Schalter für den Homing Prozess, welche einfache Kontaktflächen darstellen, eigebaut und mit dem Mikrocontroller verbunden.
Zum Schluss war es noch nötig, die endgültige Schaltung zu Verlöten und durch den Einsatz eines Spannungsteilers das Labornetzteil zu ersetzen.
\end{itemize}
Die folgende Grafik zeigt die Schematische Darstellung der Elektronik des Edubot Modells wie sie schlussendlich bei der Abgabe dieses Projektes verbaut war. Bei der Erstellung der Grafik wurde zu gunsten der Übersichtlichkeit nicht darauf geachtet dass sich die Anschlüsse an den der Realität entsprechendenden Seiten der Bauteile befinden, die Farbzuordnungen der Verbindungen entsprechen jedoch jenen der Tatsächlich verbauten Hardware.
