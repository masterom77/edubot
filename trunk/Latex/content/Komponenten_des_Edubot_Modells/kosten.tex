\subsection{Kostenaufstellung}
Die Kosten für den Bau des Edubot Modells wurden zum Großteil von der HTL Grieskirchen übernommen, da das fertige Modell mit Projektabschluss auch der Schule übergeben wurde. Beim Bau des Roboters entstanden folgende Kosten:

\begin{table}[H]
\begin{tabular}{|p{11cm}|p{3cm}|}
\hline 
\textbf{Verwendungszweck} & \textbf{Kosten}\\
\hline
1 Schrittmotoren vom Typ Nanotech ST2818M1006 & 34,60\euro{}\\
\hline
1 Schrittmotoren vom Typ Nanotech ST2818L1006 & 37,10\euro{}\\
\hline
2 Schrittmotorsteuerungen vom Typ Nanotech SMC-11 á 29,70\euro{} & 59,40\euro{}\\
\hline
4 Kugellager á 4,50\euro{} & 18\euro{}\\
\hline
1 Mikrocontroller vom Typ GHI Embedded Master Breakout Board v1.0 (gebraucht) & 30\euro{}\\ 
\hline
2qm Dreischichtplatte & 10\euro{}\\ 
\hline
Schrauben und sonstige Kleinteile & 10\euro{}\\ 
\hline
Ausgaben für Fehlkäufe und Verschleiß & 60\euro{}\\ 
\hline
\end{tabular}
\caption{Kosten des Edubot Modells}
\end{table}