\subsection{Allgemein}
Durch eine Änderung an der Anforderungen während der Erstellung dieses Projektes kam es dazu dass wir zusätzlich zum Bau des Edubot-Modells die Planung eines größeren, leistungsfähigeren Modells aufnahmen. Da wir zeitlich jedoch sehr eingeschränkt waren und sich die Aufbringung der zusätzlichen finanziellen Mittel ebenfalls als problematisch darstellte, fiel nach der ersten Entwicklungsphase der Entschluss, dieses Teilprojekt lediglich zu planen und einen Teil der für den Betrieb benötigten Software zu schreiben.

\subsubsection{Erstellte Software}
Im Rahmen der beschriebenen Planungsarbeiten wurde bereits eine Grundversion jener Software programmiert, welche später auf der SPS (Speicherprogrammierbare Steuereinheit) der Fima KEBA zum Einsatz kommen könnte.

\subsubsection{Vorhandene Hardware}
Aus Sicht der Hardware waren bei Abschluss dieses Projektes die von der Firma KEBA zur Verfügung gestellten SPS Module, die Motoren, sowie die benötigten Motorsteuerungen vorhanden. Zusätzlich wurde ein den Anforderungen eines etwa 2 Meter langen Roboterarms entsprechendes Aluprofil zur Verwendung als Achse gekauft. Ebenfalls vorhanden ist eine Aluplatte, aus welcher die benötigten Teile für die Gelenke gefräst werden können.

Die für die Herstellung der Teile des Gelenks benötigten CNC Programme wurden ebenfalls im Rahmen dieses Projekts geschrieben, gemeinsam mit der restlichen Software abgegeben und können von Herrn Dipl. Ing. Andreas Sperrer zur Verfügung gestellt werden.