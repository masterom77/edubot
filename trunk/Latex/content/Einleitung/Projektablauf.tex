\subsection{Projektablauf}
Ursprünglich sollte es sich bei diesem Projekt um eine Web-Applikation im Rahmen des Projektes Smartbow handeln. Aber als die Schüler im Juli 2011 ihre Arbeit in der MKWe begannen wurde ein neues Thema für die Diplomarbeit ausgewählt. Aus der Web-Applikation wurde eine Android-Applikation. Da das Diplomprojekt wie auch das Projekt Smartbow der MKWe mit SCRUM entwickelt wird, wurde nur ein sehr grober Backlog ausgearbeitet und die Arbeit sofort begonnen. Die Kernpunkte waren die Synchronisierung der Daten, die Tierverwaltung und eine frühe Version der Gesundheitsverwaltung.\\
In den folgenden 8 Wochen wurde Vollzeit mit den Entwicklern der MKWe gearbeitet. Wie bei SCRUM wurde in Iterationen, sogenannten Sprints gearbeitet. Die Sprints des Diplomprojektes waren unabhängig von denen des Hauptprojektes und wurden auf eine Dauer von 2 Wochen festgelegt. Wie in SCRUM vorgesehen war es stets von großer Bedeutung nach jedem Sprint eine lauffähige und fehlerfreie Version des Projektes zu haben. \\
Es wurden oft User Stories (Arbeitspakete in SCRUM) hinzugefügt oder entfernt, wenn sich die Anforderungen änderten. Dies geschah meist durch neue Testberichte von richtigen Landwirtschaftsbetrieben die das Projekt bereits verwendeten. Zu diesem Zeitpunkt gab es einige Veränderungen des Desktop-Client sowie des Servers. An diese Änderungen musste sich der mobile Client anpassen. Die wohl größte Änderung war die komplette Reversion der Gesunheitsverwaltung, wodurch ein großer Teil der Applikation unbrauchbar wurde und deswegen entfernt wurde. \\
Im  August 2011 wurde Bernhard Pflug zum Leiter des Diplomprojektes und arbeitete nun direkt mit dem Team zusammen. Nach den Ferien wurde von den Schülern meist von zu Hause aus gearbeitet, aber es wurde mit der MKWe ausgemacht das die Schüler einmal pro Woche in die Firma kommen. Da sich das Team jetzt nur noch selten traf wurde das Bugreporting mit Bugzilla eingeführt, so konnte ein Entwickler einen Fehler eintragen und alle anderen wussten darüber Bescheid.\\
Aufgrund der enormen Größe die das Projekt zu diesem Zeitpunkt bereits angenommen hatte, wurde ein eigenes Testsystem entwickelt, welches die Applikation völlig durchtestet. Dieses System wurde mit dem zentralen Build-Server der MKWe verknüpft. \\
Nachdem die Gesundheitsverwaltung beim Desktop-Client fertiggestellt wurde, konnte auch die Entwicklung dieses Features beim mobilen Client begonnen werden. Auch dabei wurde wieder viel Wert auf Testberichte gelegt. \\
Am 29.04.2012 wurde die erste Version des gesamten Projektes Smartbow released und damit auch das Diplomprojekt. Der Desktop-Client und der Server können über die MKWe Seite heruntergeladen werden und der Android-Client ist im Play Store verfügbar.
