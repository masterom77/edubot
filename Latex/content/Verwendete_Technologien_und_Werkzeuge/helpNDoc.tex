\subsection{HelpNDoc}
%http://www.helpndoc.com/
\subsubsection{Allgemein}
Die Software HelpNDoc stellt alle Funktionen zur Verfügung die zum Erstellen eines Handbuches Nötig sind. Entwickelt und zur Verfügung gestellt wird dieses Werkzeug von der Firma IBE Software. Die Verwendung ist für den persönlichen Gebrauch gratis, es werden jedoch in die erstellten Handbücher Hinweise geschrieben dass eine Proberversion verwendet wurde. Da es sich bei unserem Projekt um kein kommerzielles Projekt handelt, reichte diese Probeversion.
HelpNDoc unterstütz den Export von geschriebenen Handbüchern in verschiedenen Formaten wie etwa HTML, CHM, PDF und DOC. 

\subsubsection{Alternativen}
Für das schreiben von Handbüchern gibt es diverse kleinere Softwareangebote von verschiedenen Herstellern, die in ihrem Funktionsumfang dem von HelpNDoc ähneln. Da HelpNDoc auf den ersten Blick sehr Benutzerfreundlich schien und zudem die Ausgabe als HTLM unterstützt, fiel die Wahl sehr schnell auf diese Anwendung.

\subsubsectoin{Einsatz im Projekt}
In unserem Projekt kam die Software ausschließlich zur Erstellung des in der Anwendung vorhandenen Handbuchs zum Einsatz.