\subsection{Microsoft Expression Blend}
\subsubsection{Allgemein}
Eine  Software die von Microsoft stammt und vor allem für die Entwicklung von WPF und Silverlight Anwendungen konzipiert wurde ist Microsoft Expression Blend. Diese als IDE konzipierte Software  bietet vor allem für die Erstellung von Designs und Benutzeroberflächen zahlreiche Funktionen welche sie von Standardentwicklungsumgebungen wie Microsoft Visual Studio unterscheidet. So wird etwa der Import von Grafischen Elementen aus Produkten von Adobe erleichtert und es sind verschiedene Konvertierungsfunktionen, für Graphische Elemente vorhanden. Die Entwicklungsumgebung verfügt außerdem über eine Funktion zur schnellen Erstellung von Design Prototypen, diese Funktion wird als eine der wichtigsten Erweiterungen beworben.

In unserem Projekt wurde Microsoft Expression Blend lediglich zur Konvertierung der in Google SketchUp erstellten Modelle in ein von Visual Studio verarbeitbares Format verwendet. Diese Konvertierungsfunktion ist bereits in der Demoversion verfügbar, wodurch die Beschaffung einer Lizenz umgangen werden konnte. Für die genannte Aufgabe nutzten wir Version 4 Software.

\subsubsection{Alternativen}
Als Alternative zu Microsoft Expression Blend wäre vor allem die von der Firma WPFGraphics vertriebene Software Viewer3ds in Frage gekommen. Diese Lösung hätte eine wesentlich Komfortablere Konvertierung ermöglicht und die übernahme von Texturen erlaubt, war jedoch aus Lizenztechnischen Gründen schlichtweg zu Teuer.