\subsection{KeStudio}
\subsubsection{Allgemein}
Mit KeStudio stellt die Firma KEBA eine Entwicklungsumgebung zur Entwicklung von Software für Steuerungssysteme zur Verfügung. Dabei verwendet KeStudio die Programmiersoftware CoDeSys und bietet zusätzlich umfangreiche Unterstützung für die hauseigenen Systeme zur Bahnberechnung die unter dem Namen KeMotion zur Verfügung stehen. KeStudio hat zudem einen einfach zu bedienenden Konfigurationsmanager, der es ermöglicht eine Verbindung mit der gewünschten SPS herzustellen und alle relevanten Einstellungen an der SPS vorzunehmen.
\subsubsection{CoDeSys}
Die von KeStudio zur SPS Programmierung verwendete Programmiersoftware heißt CoDeSys. CoDeSys implementiert den Standard IEC 61131-3 welcher genau definiert welche Voraussetzungen eine Programmiersoftware für Speicherprogrammierbare Steuerungen erfüllen muss. Unter anderem definiert dieser Standard fünf Arten von Programmiersprachen, wobei sowohl textbasierte Sprachen wie IL und ST, als auch grafikbasierte Sprachen wie FBD, SFC und LD vorgegeben werden. In allen Sprachen können Funktionen und Funktionsblöcke verwendet werden, die in einer der anderen Sprachen geschrieben wurden. Es ist den SPS Herstellern auch möglich Funktionsbibliotheken ohne Quelltext mitzugeben, welche ebenfalls aus allen Sprachen aufgerufen werden können.
Grundsätzlich ist es je nach Leistungsfähigkeit der SPS nicht nötig sämtliche oben genannte Sprachen zu unterstützen. Die über KeStudio programmierbaren SPS Systeme der Firma KEBA unterstützen jedoch alle diese Sprachen, sowie zwei zusätzliche Sprachen mit verbessertem Funktionsumfang.\footcite[vgl.][]{codesys}
\subsubsection{Alternativen}
Da es sich bei den in der Schule vorhandenen SPS Systemen um Systeme der Firma KEBA handelt, diese optimal mit KeStudio kommunizieren und diese Software zudem der Schule zur Verfügung steht, wurde in diesem Zusammenhang nicht nach Alternativen gesucht.
\subsubsection{Einsatz im Projekt}
In unserem Projekt kam KeStudio ausschließlich zur Programmierung der SPS Systeme eingesetzt.
