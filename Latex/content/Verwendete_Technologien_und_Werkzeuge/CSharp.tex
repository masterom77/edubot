\subsection{C\#}
C\# ist eine von Microsoft zur Verfügung gestellte Programmiersprache. Sie ist den objektorientierten Hochsprachen zuzuordnen und befindet sich derzeit in Version 4.0. In C\# vereinen sich verschiedene Konzepte der Programmiersprachen Java, C++, Haskell, C und Delphi. Besonders für Entwickler die mit der Entwicklung von Java Applikationen vertraut sind fällt der Umstieg auf C\# vergleichsweise leicht, da sich die Sprachen sowohl in der Syntax, als auch in den zur Verfügung gestellten Funktionen stark ähneln.

C\# wurde vor allem  für die Entwicklung von Pragrammen auf Basis des .Net Frameworks zu erstellen, es wird aber auch die Entwicklung von COM Komponenten ermöglicht. COM ist ein  Vorläufer des .Net Konzepts der die Wiederverwendung von Programmcode erleichtern sollte.
Eine Auflistung und Erklärung der wichtigsten Sprachfunktionen und Konzepte würde den Rahmen dieser Arbeit sprengen und wird hier damit nicht angeführt.

Die Hauptgründe warum bei der Entwicklung unseres Projekts C\# zum Einsatz kam sind folgende:
\begin{itemize}
\item Gute Erfahrungswerte\\

\item Gute Unterstützung in Visual Studio\\

\item Vertrautheit mit der Sprachsyntax\\

\item Vielfältige Einsetzbarkeit\\
\end{itemize}