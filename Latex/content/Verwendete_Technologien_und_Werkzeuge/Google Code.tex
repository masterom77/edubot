\subsection{Google Code}
\subsubsection{Allgemein}
%http://code.google.com/p/support/wiki/GettingStarted#Getting_Started
Google Code ist eine, vom Unternehmen Google Inc., zur Verfügung gestellte Online-Plattform zur Verwaltung von Open-Source Projekten. Die von Benutzern erstellten Projekte können von anderen Personen über die Website gesucht werden. 
Google Code stellt dem Benutzer zusätzlich einige Nützliche Werkzeuge wie beispielsweise ein Dokumentationstool oder einen integrierte Aufgabenverwaltung zur Verfügung. Es ist auch möglich fertig compilierte Software, beispielsweise als beta version oder dergleichen direkt über Google Code zu verteilen. 

Die wohl wichtigste Funktion von Google Code ist es, für das gehostete Projekt ein "'repository"' für die Versionsverwaltung zur Verfügung zu stellen. Das heißt dass je nach verwendeter Technologie (Subversion, Mercurial und GIT werden unterstützt), mithilfe entsprechender Client-Anwendungen (siehe Kapitel Subversion) das Projekt über den Google Code Server auf jedem Entwickler PC aktuell gehalten werden kann.

\subsubsection{Verwendung im Projekt}
In unserem Projekt kam Google Code ausschließlich zur Versionsverwaltung zum Einsatz, andere Funktionen des Services wurden nicht genuzt. 

