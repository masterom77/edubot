\subsection{Werkzeuge für die Schriftliche Arbeit}
\subsubsection{LaTex}
LaTeX ist ein Softwarepaket, welches die Benutzung von Tex mittels Makros vereinfacht. Tex ist ein Textverarbeitungsprogramm das nicht nach dem What-you-see-is-what-you-get Prinzip arbeitet, stattdessen wird Formatierung mittels Befehle, so zusagen programmiert. Dies hat den Vorteil, dass das Layout sehr sauber bleibt und sich nicht versehentlich verändern lässt. Daher wird es sehr gerne für umfangreiche Arbeiten wie Diplomarbeiten und Dissertationen verwendet. Wir haben mit Latex die Erfahrung gemacht, dass man zwar länger zum Konfigurieren braucht, jedoch der Schreibprozess wesentlich produktiver durchführbar ist.\footcite[vgl.][]{latex}
\subparagraph{TexMaker}
TexMaker ist ein plattformunabhängiger LaTex Editor und läuft unter Linux,Mac OS X und Windows. Er stellt sehr viele Funktionen wie Autokorrektur und Autovervollständigung bereit, jedoch musste man jede einzelne 
LaTeXdatei einzeln öffnen und zum Bauen auf die Hauptdatei zurückspringen. Dies war der Grund, dass wir begannen eine Alternative zu suchen.\footcite[vgl.][]{texmaker}
\subparagraph{Texpad}
TexPad ist ein LaTeX Editor für den Mac und wird derzeit noch im Mac App Store angeboten. Dieser ist sehr einfach zu verwenden, da nur die Hauptlatexdatei ausgewählt werden muss. Das Programm lädt anschließend alle referenzierten Dateien automatisch. Außerdem kann man durch Klicken der Textpassagen im erstellten PDF, direkt auf das richtige Latexfile springen.
Diese Funktionen erleichtern und beschleunigen den Schreibprozess sehr, daher haben wir uns für diesen Editor entschieden.\footcite[vgl.][]{texpad}


 
\subsubsection{Google Code}

Google Code ist ein Dienst von Google, der es dem Entwickler ermöglicht seine Projekte im Internet zu veröffentlichen. Der Entwickler hat die Möglichkeit seinen Source Code hochzuladen und mittels SVN,Mercurial oder GIT zu Versionieren. Außerdem hat er die Möglichkeit Wiki-Einträge zur Dokumentation seines Projekts zu erstellen. In unserem Fall verwendeten wir Google Code um zu zweit an der schriftlichen Arbeit arbeiten zu können.\footcite[vgl.][]{googlecode}