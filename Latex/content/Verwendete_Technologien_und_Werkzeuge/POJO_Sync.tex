
\subsection{POJO Sync}
Bei POJO Sync handelt es sich um ein Tool zur Sychronisierung von Java-Objekten zwischen einem Server und mehreren Clients. Es wurde im Zuge des Smartbow Projekts von Martin Oberhauser entwickelt und als quelloffenes Tool freigeben. 
POJO steht für Plain Old Java Objects, dies wiederum bedeutet, dass es sich um besonders einfach gehaltene Objekte handelt. Diese Objekte beinhalten meist nur Attribute und deren Zugriffsmethoden zum Setzen und Auslesen der Werte. In unserem Fall sind das Objekte wie z.B. das Tier inklusive deren einzelnen Attribute.

\subparagraph{Der Synchronisierungsprozess von POJO Sync}

Um die Synchronisierung zu starten muss der Client sich am Server anmelden. Ist diese Anmeldung erfolgreich, wird eine Socketverbindung zwischen den beiden Geräten erstellt. Um nun die Synchronisierung zu starten muss der Client die \textit{synchronize} Methode aufrufen. Der Client bekommt nun die Versionsnummer des letzten Standes vom Server zugeschickt. Ist diese höher als die lokale Versionsnummer, muss der Client aktualisiert werden. Der Client fordert dazu den Server auf, ihm die fehlenden Daten zu schicken. Ist dies abgeschlossen fragt der Server den Client, ob auch er neue Daten hat. Diese werden vom Client in der Modifikationstabelle gespeichert. Sind Datensätze in dieser Tabelle bedeutet das, dass neue Daten vorhanden sind und der Server fordert den Client auf, ihm diese Daten zu schicken. Falls keine Fehler entstanden sind, erstellt der Server eine neue Versionsnummer und schickt sie wieder zurück zum Client. Nun löscht der Client alle Modifkationen aus der Tabelle und setzt die Versionsnummer auf den neuen Wert. Jetzt sind beide Geräte wieder am selben Stand und der Synchronisierungsprozess ist beendet.

\begin{figure}[htbp]
\centering
    \includegraphics[width=15cm]{images/synchronisierung}
\caption{POJO Synchronisierungsprozess}
\end{figure}
\newpage

\subparagraph{Der ContentStorage und Mappings}
Um diesen Sychronisierungsprozess zu ermöglichen, muss ein ContentStorage implementiert werden. Diese Klasse ist dafür zuständig, dass alle Daten von der Synchronisierung richtig ausgelesen und geschrieben werden können.
Dieser Storage benutzt unter anderem auch sogenannte Mappings, die in der Mappings Tabelle gespeichert werden. Diese Tabelle wurde eingeführt, da es vorkommen kann, dass die lokale Id eines Objektes nicht mit der Id vom Server übereinstimmt. Um dieses Problem zu lösen werden beide Ids in dieser Mappings Tabelle gespeichert und man kann mittels einer Id die zugehörige Id herausfinden. Zusätzlich wird noch der Type gespeichert, dieser wird von der Synchronisierung zur eindeutigen Identifikation genutzt.


\begin{figure}[H]
\centering
\includegraphics[width=4cm]{images/mappings}
\caption{Struktur der Mappings Tabelle}
\end{figure}

\subparagraph{Der Tabellen Modifcations und LatestRevision}
Wie schon im Synchronisierungprozess erwähnt, fragt die Station den Client um die Versionsnummer seines letzten Standes. Diese LatestRevision wird in die Tabelle latestrevision gespeichert. Diese Tabelle beinhaltet nur diesen einen Datensatz, der bei jeder erfolgreichen Synchronisierung verändert wird.
Ebenfalls wurden die Modifikationen im Synchronisierungsprozess erwähnt. Solch eine Modifikation wird bei der Erstellung oder Änderung eines Objekt erzeugt. Dazu vergleicht der Synchronizer, das alte Objekt mit dem neuen Objekt und erkennt dabei die einzelnen Veränderungen. Jede einzelne Veränderung wird dann mit dem Objekttyp,der Id,dem Attributnamen und dem Wert in die Modifcations Tabelle gespeichert. Bei einem neu erstellten Objekt wird mit einem leeren Objekt verglichen und zusätzlich ein "isNew=true" als Modifkation gespeichert. Diese Modifkationen werden vom Synchronisierungsprozess ausgelesen und auf der Station durchgeführt. Wenn dieser Prozess erfolgreich war wird die Modifications Tabelle geleert.


\begin{figure}[H]
  \centering
  \begin{minipage}[t]{7 cm}
    \centering
    \includegraphics[width=4cm]{images/latestrevision} 
    \caption{Struktur der LatestRevision Tabelle}
  \end{minipage}
  \hspace{0.5cm}
  \begin{minipage}[t]{7 cm}
	\centering
	\includegraphics[width=4cm]{images/modifications}  
    \caption{Struktur der Modifications Tabelle}
  \end{minipage}
  
\end{figure}









 











