
\subsection{Kommando-System}

\subsubsection{Aufgaben}
Dem Roboter sollen schnell und einfach Befehle mitgeteilt werden können, weshalb hierfür ein gut funktionierendes System benötigt wird. Befehle sollten direkt der Edubot-Klasse übergeben werden können, welche sich um die Ausführung letzterer kümmert. Zukünftig sollen auch neue Befehle hinzufügt werden können um die API flexibel und ausbaufähig zu halten.

\subsubsection{Aufbau}
Das Kommando-System ist relativ simpel aufgebaut und beruht auf einer ähnlichen Architektur wie das Adapter-System. Als Fundament des Systems dient das Interface \textit{ICommand} verwendet. Die individuellen Commands müssen dieses Interface implementieren und sollen sich selbst um die durchzuführenden Aktionen kümmern, weshalb sie die vordefinierte Methode \textit{Execute} implementieren müssen. Weiters soll jedes Command selbstständig prüfen ob seine Ausführung zu diesem Zeitpunkt zulässig ist, weshalb das Command innerhalb der \textit{Execute}-Methode Zugriff auf das \textit{State}-Property des jeweiligen Adapters benötigt. Auch für zukünftige entwickelte Commands wird vor deren eigentlicher Durchführung eine Validierung des Adapter-Zustands empfohlen.

\subsubsection{Umsetzung}
\textbf{ICommand}
\newline
Wie bereits im Aufbau angemerkt, beruht das gesamte System auf dem Interface \textit{IAdapter}, welches den Grundaufbau eines Befehls festlegt. Dieser Aufbau besteht lediglich aus der im Anschluss beschriebenen Methode.
\begin{itemize}
\item \textbf{Execute}
\newline
Bei Aufruf dieser Methode werden die Voraussetzungen für das Ausführen des Befehls geprüft und anschließend von der Art des Befehls abhängigen Aktionen durchführt. Die Methode erhält als Parameter ein Objekt vom Typ \textit{IAdapter}, welches den ausführenden Adapter enthält. Durch diesen Adapter besteht erst die Möglichkeit eine Zustandsvalidierung durchzuführen. Weiters wird empfohlen die zum Command gehörige Adapter-Methode in einem eigenen Thread zu starten um den Adapter selbst nicht unnötig lange zu blockieren.
\end{itemize}
\newpage
Die folgende Auflistung beschreibt alle, im Zuge der Diplomarbeit entwickelten, Befehle:

\textbf{StartCommand}
\newline
Dieser Befehl dient zum Einschalten und Initialisieren des Roboters und stellt im Regelfall den ersten Befehl dar, der an den Roboter geschickt wird. Die \textit{Execute}-Methode ist dabei folgendermaßen implementiert:
\begin{itemize}
\item \textbf{Execute}
\newline
Bei der Zustandsvalidierung wird geprüft ob sich der Adapter im \textit{SHUTDOWN}-Zustand befindet. Sollte sich der Adapter in einem anderen Zustand befinden wird eine \textit{IllegalStateException} ausgelöst und Ausführung des Befehls abgebrochen. Ist dies nicht der Fall so wird das \textit{State}-Property auf \textit{HOMING} gesetzt und das \textit{OnHoming}-Event ausgelöst (mehr dazu im Kapitel Event-System). Zuletzt wird noch ein Thread gestartet, der die \textit{Start}-Methode des Adapter ausführt.
\end{itemize}

\textbf{ShutdownCommand}
\newline
Dieser Befehl dient zum Herunterfahren des Roboters und stellt im Regelfall den letzten Befehl dar, der an den Roboter geschickt wird. Die \textit{Execute}-Methode ist dabei folgendermaßen implementiert:
\begin{itemize}
\item \textbf{Execute}
\newline
Bei der Zustandsvalidierung wird geprüft ob sich der Adapter im \textit{READY}-Zustand befindet. Sollte sich der Adapter in einem anderen Zustand befinden wird eine \textit{IllegalStateException} ausgelöst und Ausführung des Befehls abgebrochen. Ist dies nicht der Fall so wird das \textit{State}-Property auf \textit{SHUTTING\_DOWN} gesetzt und das \textit{OnShuttingDown}-Event ausgelöst. Zuletzt wird noch ein Thread gestartet, der die \textit{Shutdown}-Methode des Adapter ausführt.
\end{itemize}

\textbf{MVSCommand}
\newline
Dieser Befehl dient zum Verfahren einer linearen Bewegung durch den Roboter. Die \textit{Execute}-Methode ist dabei folgendermaßen implementiert:
\begin{itemize}
\item \textbf{Execute}
\newline
Bei der Zustandsvalidierung wird geprüft ob sich der Adapter im \textit{READY}-Zustand befindet. Sollte sich der Adapter in einem anderen Zustand befinden wird eine \textit{IllegalStateException} ausgelöst und Ausführung des Befehls abgebrochen. Ist dies nicht der Fall so wird das \textit{RequiresPrecalculation}-Property des Adapters geprüft. Ist dieser Wert auf \textit{true} gesetzt, so wird der Pfad zum Zielpunkt und die daraus resultierenden Winkelschritte mit Hilfe der \textit{LinearInterpolation}-Klasse berechnet. Anschließend wird die \textit{SetInterpolationResult}-Methode des Adapters aufgerufen, der das Interpolationergebnis in Form eines \textit{InterpolationResult}-Objekts übergeben wird. Im nächsten Schritt wird das \textit{State}-Property auf \textit{MOVING} gesetzt und das \textit{OnMovementStarted}-Event ausgelöst. Zuletzt wird noch ein Thread gestartet, der die \textit{MoveStraight}-Methode des Adapter ausführt.
\end{itemize}

\textbf{MVCCommand}
\newline
Dieser Befehl dient zum Verfahren einer zirkularen Bewegung durch den Roboter. Die \textit{Execute}-Methode ist dabei folgendermaßen implementiert:
\begin{itemize}
\item \textbf{Execute}
\newline
Bei der Zustandsvalidierung wird geprüft ob sich der Adapter im \textit{READY}-Zustand befindet. Sollte sich der Adapter in einem anderen Zustand befinden wird eine \textit{IllegalStateException} ausgelöst und Ausführung des Befehls abgebrochen. Ist dies nicht der Fall so wird das \textit{RequiresPrecalculation}-Property des Adapters geprüft. Ist dieser Wert auf \textit{true} gesetzt, so wird der Pfad zum Zielpunkt und die daraus resultierenden Winkelschritte mit Hilfe der  \textit{CircularInterpolation}-Klasse berechnet. Anschließend wird die \textit{SetInterpolationResult}-Methode des Adapters aufgerufen, der das Interpolationergebnis in Form eines \textit{InterpolationResult}-Objekts übergeben wird. Im nächsten Schritt wird das \textit{State}-Property auf \textit{MOVING} gesetzt und das \textit{OnMovementStarted}-Event ausgelöst. Zuletzt wird noch ein Thread gestartet, der die \textit{MoveCircular}-Methode des Adapter ausführt.
\end{itemize}

\textbf{UseToolCommand}
\newline
Dieser Befehl dient zum Aktivieren oder Deaktiveren des verwendeten Werkzeugs. Die \textit{Execute}-Methode ist dabei folgendermaßen implementiert:
\begin{itemize}
\item \textbf{Execute}
\newline
Bei der Zustandsvalidierung wird geprüft ob sich der Adapter im \textit{READY}-Zustand befindet. Sollte sich der Adapter in einem anderen Zustand befinden wird eine \textit{IllegalStateException} ausgelöst und Ausführung des Befehls abgebrochen. Anschließend wird das \textit{State}-Property auf \textit{MOVING} gesetzt und das \textit{OnToolUsed}-Event ausgelöst. Zuletzt wird noch ein Thread gestartet, der die \textit{UseTool}-Methode des Adapter ausführt. 
\end{itemize}

\textbf{AbortCommand}
\newline
Dieser Befehl dient zum sofortigen Abbruch jeglicher Aktionen die der Roboter im Moment durchführt. Die \textit{Execute}-Methode ist dabei folgendermaßen implementiert:
\begin{itemize}
\item \textbf{Execute}
\newline
Da ein Abbruch aus Sicherheitsgründen jederzeit möglich sein muss, wird bei diesem Befehl auf etwaige Zustandsvalidierung verzichtet. Anschließend wird die Warteschlange des Roboters geleert um die Gefahr durch nachfolgende Befehle zu eliminieren und das \textit{OnAbort}-Event ausgelöst. Zuletzt wird noch ein Thread gestartet, der die \textit{Abort}-Methode des Adapter ausführt. 
\end{itemize}