\subsection{Projektumgebung}
Das Projekt wurde für HTBLA Grieskirchen für Demonstrationen und Lehrzwecke entwickelt, wobei unser Betreuungslehrer Dipl-Ing. Andreas Sperrer gleichzeitig die Rolle des Auftraggebers übernahm. 

\subsubsection{Projektteam}
Das Projektteam bestand aus Philipp Stelzer und David Maier sowie aus unserem Betreuungslehrer Dipl-Ing. Andreas Sperrer. Die Projektleitung wurde formell von Philipp Stelzer übernommen, unter Rücksichtnahme auf die Tatsache dass als Entwicklungsmethode SCRUM verwendet wurde. Aus diesem Grund war die Aufteilung der organisatorischen Aufgaben eher ausgewogen.

\begin{itemize}

\item \textbf{Philipp Stelzer}\\
Als formeller Projektleiter des Projektes umfassten seine Aufgaben folgendes:
\begin{itemize}
\item Projektorganisation
\item Auswahl der Hardware
\item Planung der API
\item Entwicklung der API
\begin{itemize}
\item Entwicklung Befehlsübergabe
\item Entwicklung Edubot-Klasse
\item Entwicklung Adapter-System
\begin{itemize}
\item Entwicklung EdubotAdapter
\item Entwicklung KebaAdapter
\item Entwicklung VirtualAdapter
\end{itemize}
\item Entwicklung Listener-System
\item Entwicklung Zustands-System
\item Entwicklung Event-System
\item Entwicklung Kommando-System
\item Entwicklung Kinematik und Interpolation
\end{itemize}
\item Erstellung von Grafiken für die Endanwendung
\item Entwicklung der Endanwendung
\begin{itemize}
\item Entwicklung Benutzeroberfläche (Dateneingabe)
\item Entwicklung Standardbefehle
\item Entwicklung Kommandozeile
\item Entwicklung Zeichenfläche
\item Entwicklung Konfiguration
\end{itemize}
\end{itemize}

In der schriftlichen Arbeit erstellte er folgende Bereiche:
\begin{itemize}
\item Einleitung
\begin{itemize}
\item Innovation
\item Projektumgebung
\end{itemize}
\item Verwendete Technologien \& Werkzeuge
\begin{itemize}
\item Windows Presentation Foundation
\end{itemize}
\item Komponenten der API
\begin{itemize}
\item Die Edubot-Klasse
\item Adapter-System
\item Listener-System
\item Zustands-System
\item Event-System
\item Kinematik
\item Lineare Interpolation
\item Zirkulare Interpolation
\end{itemize}
\end{itemize}
     
\item \textbf{David Maier}\\
Als einer der Hauptentwickler des Projektes umfassten seine Aufgaben folgendes:
\begin{itemize}
\item Studien zur Hardwareauswahl
\item Planung der Endanwendung
\item Entwicklung der Endanwendung
\begin{itemize}
\item Entwicklung der Visualisierungen
\item Einbindung des Hilfesystems
\end{itemize}
\item Entwicklung an der KEBA SPS
\item Entwicklung einer Netzwerkschnittstelle 
\item Entwicklung der Komunikation mit den Motoren
\item Einbindung von KeMotion für die Interpolation
\item Entwicklung am Microcontroller von GHI
\item Planung der Holzkonstruktion von Edubot
\item Bau der Holzkonstruktion von Edubot
\item Studien zur Alukonstruktion für das KEBA System
\item Entwicklung der CNC Programme
\item Planung der weiteren Schritte für den Bau
\end{itemize}

In der schriftlichen Arbeit erstellte er folgende Bereiche:
\begin{itemize}
\item Einleitung
\begin{itemize}
\item Aufgabenstellung
\item Projektablauf
\item Persönliche Erfahrung
\end{itemize}
\item Verwendete Technologien \& Werkzeuge
\begin{itemize}
\item C\#
\item Microsoft Visual Studio 2010
\item Microsoft Expression Blend 4.0
\item Latex
\end{itemize}
\item Komponenten der Hardware
\end{itemize}
     

\end{itemize}
   
\subsubsection{Auftraggeber}
Den Anstoß und Auftrag für die Entwicklung von Edubot kam von Herrn \textbf{Dipl.-Ing. Andreas Sperrer}, welcher das Fach “Prozessregelung mit Laborübungen” in den 3. Klassen unterrichtet.

\subsubsection{Betreuer}
Die Rolle des Betreuers wurde ebenfalls von Herrn Dipl.-Ing. Andreas Sperrer übernommen, wodurch er zwei Rollen in diesem Projekt einnimmt. Zu seinen Aufgaben zählte die Beratung in Hinsicht auf die Auswahl der Hardware, sowie Hilfestellung bei mathematischen Problemen. Weiters stellte er uns zahlreiche Materialien für den Roboterbau, sowie Bücher aus seinem eigenen Sortiment zur Verfügung.

