\subsection{Projektumgebung}
Das Projekt wurde für und mit der Firma \textit{MKW electronics} durchgeführt. Da es sich um ein Unterprojekt des Projekts \textit{SmartBow} handelt, wurde der ganze Entwicklungsprozess stark mit dem der Angestellten von \textit{MKWe} kombiniert.

\subsubsection{Projektteam}
Aufgrund der engen Zusammenarbeit mit den Mitarbeitern von \textit{MKWe} werden diese auch zum Projektteam hinzugezählt. Somit bestand das gesamte Team aus 8 Entwicklern, darunter 2 HTL-Schülern.

\begin{itemize}

\item \textbf{Tobias Geibinger}\\
Als einer der Hauptentwickler des Projektes umfassten seine Aufgaben folgendes:
\begin{itemize}
\item Planung und Konzeptentwicklung
\item Entwicklung von UI-Elementen
\item Entwicklung der Datenbankanbindung
\item Entwicklung Tierverwaltung
\begin{itemize}
\item Erstellung UI
\end{itemize}
\item Entwicklung Gesundheitsverwaltung
\begin{itemize}
\item Erstellung UI
\end{itemize}
\item Entwicklung Hauptfenster
\item Entwicklung des Testrunner
\item Qualitätssicherung
\end{itemize}

In der schriftlichen Arbeit erstellte er folgende Bereiche:
\begin{itemize}
\item Einleitung
\begin{itemize}
\item Aufgabenstellung
\item Projektumgebung
\item Projektziele
\item Projektablauf
\item Innovation
\item Persönliche Erfahrung
\end{itemize}
\item Komponenten
\begin{itemize}
\item Synchronisierung
\item Tierverwaltung
\item Gesundheitsverwaltung
\item Testrunner
\end{itemize}
\end{itemize}
     
\item \textbf{Mathias Aichinger}\\
Als einer der Hauptentwickler des Projektes umfassten seine Aufgaben folgendes:
\begin{itemize}
\item Planung und Konzeptentwicklung
\item Entwicklung der Datenbankanbindung
\item Entwicklung Serveranmeldung
\item Entwicklung Synchronisierung
\item Entwicklung Tierverwaltung
\begin{itemize}
\item Erstellung Datenanbindung
\end{itemize}
\item Entwicklung Gesundheitsverwaltung
\begin{itemize}
\item Erstellung Datenanbindung
\end{itemize}
\item Qualitätssicherung
\end{itemize}

In der schriftlichen Arbeit erstellte er folgende Bereiche:
\begin{itemize}
\item Einleitung
\begin{itemize}
\item Persönliche Erfahrung
\end{itemize}
\item Verwendete Technologien \& Werkzeuge
\item Komponenten
\begin{itemize}
\item Serveranmeldung
\item Hauptfenster
\end{itemize}
\end{itemize}
     
\item \textbf{Bernhard Pflug}\\
Bernhard Pflug ist ein Mitarbeiter der \textit{MKWe} und war ebenfalls maßgeblich am Projekt beteiligt. Er diente den HTL-Schülern als Schnittstelle zur Firma und den Zielen des Projektes. Zu seinen Aufgaben zählte:     
\begin{itemize}
\item Planung und Konzeptentwicklung
\item Entwicklung der Suche
\item Erstellen von Code-Abstrahierungen
\item Qualitätssicherung
\end{itemize}     
     
\item \textbf{Martin Oberhauser}\\
Martin Oberhauser ist ebenfalls ein Mitarbeiter der \textit{MKWe}. Seine Aufgabe im Projekt war vorwiegend uns mit der von ihm entwickelten Synchronisierung (POJO Sync) vertraut zu machen.
     
\item \textbf{Manfred Öhlschuster}\\
Ein Angestellter der \textit{MKWe}. Seine Aufgabe im Projekt war die Mithilfe beim Einrichten des ORMLite Systems in der Applikation.     
     
\item \textbf{Marcel Otte}\\   
Ein Angestellter der \textit{MKWe}. Seine Aufgaben waren meist nicht direkt projektrelevant da er hauptsächlich am Desktop-Client und am Server arbeitete. Dennoch musste die Android-Applikation oft seinen Änderungen angepasst werden. 
     
\item \textbf{David Andlinger}\\
Ein Angestellter der \textit{MKWe}. Wie schon bei Marcel war David auch nicht direkt am Projekt beteiligt wurde aber oft über seine Arbeit am Desktop-Client befragt, wenn es zu Unklarheiten kam.

\item \textbf{Tobias Noiges}\\
Tobias Noiges programmierte im Rahmen seiner Masterarbeit eine Augmented-Reality Applikation die von ihm in die Android-Applikation integriert wurde. Im Rahmen dieser Integrierung hatte Tobias oft fragen zu unserem System.

\end{itemize}
   
\subsubsection{Auftraggeber}
\textbf{Wolfgang Auer} der \textit{General Manager} der \textit{MKWe} übernahm die Aufgabe des Auftraggebers, sowie die SCRUM-Rolle des \textit{Project Owner}. Er legte mit dem Projektteam die Ziele und User Stories fest und kontrollierte am Ende jeder Iteration ob diese wie geplant implementiert wurden.

\subsubsection{Betreuer}
Als Betreuer dieser Diplomarbeit fungierte \textbf{Dipl.Ing. Wolfgang Kaiser}. Seine Aufgaben waren die Kontrolle des Projektfortschrittes und die Beratung bei aufgetretenen Problemen. Dies funktionierte durch regelmäßige Besprechungen und die Kontrolle von wöchenentlichen Arbeitsberichten der HTL-Schüler.

