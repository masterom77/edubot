\subsection{Innovation}
Die Innovation des Projekts liegt darin, dass bisher weder ein Roboter, noch eine enstprechende Software für den Unterricht im Unterrichtsfach PRRV 3 vorhanden ist. Es wird derzeit lediglich mit Powerpoint-Präsentationen und Filmen gearbeitet, jedoch existiert keine Möglichkeit das erlernte Robotik-Wissen einzusetzen bzw. die Prinzipien der Robotik anschaulich zu präsentieren. 
\newline
Durch das Projekt Edubot werden den Schülern die soeben genannten Dinge ermöglicht und zusätzlich noch die Möglichkeit gegeben einen Roboter in eigene Programmen einzubinden. Die API wurde möglichst einfach gehalten, sodass ihre Verwendung lediglich Basiswissen zur Robotik vorraussetzt.
Daraus ergeben sich folgende innovative Aspekte:
\begin{itemize}
\item \textbf{Robotermodell}
\newline
Die Schule ist derzeit nicht in Besitz eines Roboters.
\item \textbf{API}
\newline
Da kein Roboter existiert, gibt es natürlich auch keine Anwendungsbibliothek die es ermöglicht diesen zu steuern. Schüler können Hardware lediglich über ein USB-Modul und die dazugehörige Bibliothek in ihre Anwendung einbinden. Dazu ist jedoch detailliertes Wissen über die Verkabelung nötig.
\item \textbf{Anwendung}
Beispiele aus der Robotik werden üblicherweise durch Videos beziehungsweise Bilder präsentiert. Eine gezielte Veranschaulichung der bestimmter Bewegungen ist damit nur schwer möglich.
\end{itemize}
