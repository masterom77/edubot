\subsection{Persönliche Erfahrung}
\par
\begingroup
\leftskip=1cm 
\rightskip=1cm 
\noindent
Durch die Planung und Durchführung dieses Projekts konnte ich eine große Menge an Wissen und Erfahrungen im Bereich Softwarearchitektur und Robotik sammeln. Da in vergangenen Projekten das Ziel lediglich die Entwicklung einer Endanwendung vorsah, war das Design einer wiederverwendbaren Programmbibliothek eine besondere Herausforderung für mich. Bei der Planung kamen mir immer wieder neue Ideen, von denen manche jedoch nicht im zeitlichen Rahmen der Diplomarbeit realisierbar gewesen wären. Zusätzlich zu diesem Wissen, war es für die Pfadberechnung der Roboterbewegungen nötig mein mathematisches Wissen in der Trigonometrie aufzufrischen. Nach zahlreichen Skizzen und einigen Meetings mit unserem Betreuungslehrer konnte ich das erhaltene Wissen in der API verwenden. Es war ein großartiges Erfolgserlebnis als die erste, von der API berechnete, Bewegung erfolgreich vom Roboter durchgeführt wurde. Die praktische Verwendung des agilen Vorgehensmodells SCRUM, zeigte mir wie wichtig eine konkrete Zieldefinition für jede Phase der Planung und Entwicklung ist. Da bei diesem Modell die Anforderungen laufend aktualisiert werden, traf uns die Änderung der Hardware-Anforderungen weniger hart als bei anderen Planungssystemen. Abschließend möchte ich sagen, dass mir diese Diplomarbeit gezeigt hat, wie Dinge die auf den ersten Blick kompliziert wirken bei näherer Betrachtung logisch erklärbar werden. - \textit{Philipp Stelzer}
\\\\
\\
Da sich vor diesem Projekt mein Wissen über viele der verwendeten Technologien auf ein Minimum beschränkte, war es mir durch die intensive Beschäftigung mit diesen Themen möglich einen tieferen Einblick zu erlangen. Als Beispiele hierfür sind vor allem die Berechnungen für die Kinematik des Roboters, die Programmierung des Mikrocontrollers und der SPS, sowie der Bau eines physischen Roboterarms anzuführen. Da diese Themen während unserer schulischen Ausbildung nur am Rande erwähnt wurden, ermöglichte mir dieses Projekt eine genauere Beschäftigung mit den genannten Themengebieten und damit das Sammeln von sowohl theoretischem, als auch praktischem Wissen.
Es war zudem sehr interessant zu erleben wie die Projektplanung in einem Projekt dieser Größenordnung abgewickelt werden kann. Ich hatte bereits im Vorfeld dieses Projektes im Rahmen des Pflichtpraktikums die Möglichkeit größere Projekte selbständig abzuwickeln, jedoch war hier nie eine agile Methode zur Softwareentwicklung zum Einsatz gekommen. 
\newpage
In diesem Projekt war es mir möglich selbst zu erleben welche Vorteile und welche Probleme bei der Verwendung eines solchen Werkzeuges entstehen.  - \textit{David Maier}
\par
\endgroup
