\subsection{Persönliche Erfahrung}
\par
\begingroup
\leftskip=1cm 
\rightskip=1cm 
\noindent
Durch die Planung und Durchführung dieses Projekts konnte ich eine große Menge an Wissen und Erfahrungen im Bereich Softwarearchitektur und Robotik sammeln. Da in vergangenen Projekten das Ziel lediglich die Entwicklung einer Endanwendung vorsah, war das Design einer wiederverwendbaren Programmbibliothek eine besondere Herausforderung für mich. Bei der Planung kamen mir immer wieder neue Ideen, von denen manche jedoch nicht im zeitlichen Rahmen der Diplomarbeit realisierbar gewesen wären.
Zusätzlich zu diesem Wissen, war es für die Pfadberechnung der Roboterbewegungen nötig mein mathematisches Wissen in der Trigonometrie aufzufrischen. Nach zahlreichen Skizzen und einigen Meetings mit unserem Betreuungslehrer konnte ich das erhaltene Wissen in der API verwenden. Es war ein großartiges Erfolgserlebnis als die erste, von der API berechnete, Bewegung erfolgreich vom Roboter durchgeführt wurde.
Durch dieses Projekt war es mir möglich eine Menge Erfahrungen über mein zukünftiges Berufsfeld zu sammeln. Es war sehr interressant zu sehen wie Software-Entwicklung in der Praxis funktioniert, besonders mit einem agilen Vorgehensmodell wie SCRUM. Auch wenn unsere Schule schon sehr realitätsnah ist, ist in der Praxis doch noch einiges anders. Eine weitere tolle Erfahrung war das Arbeiten in einem richtigen Team. Nicht nur auf technischer Ebene, sondern auch das soziale Umfeld. Von der technischen Seite habe ich viel über Programmierpraktiken und über neue Technologien gelernt.  Außerdem habe ich im Laufe des Projektes viel über die Wichtigkeit von konkreten Zielen gelernt und das immerwieder unerwartete Probleme auftreten können. Zusammenfassend lässt sich sagen, dass es eine tolle Erfahrung war und mir gezeigt hat, dass dieser Weg der richtige für mich ist. - \textit{Tobias Geibinger}
\\\\
Da sich vor diesem Projekt unser Wissen über viele der verwendeten Technologien auf ein Minimum beschränkte, war es uns durch die intensive Beschäftigung mit diesen Themen möglich einen Tieferen Einblick zu erlangen. Als Beispiele hierfür sind vor allem die Berechnungen für die Kinematik des Roboters, die Programmierung des Mikrocontrollers und der SPS, sowie der Bau eines physischen Roboterarms anzuführen. Da diese Themen während unserer schulischen Ausbildung nur am Rande erwähnt wurden, ermöglichte es uns dieses Projekt eine genauere Beschäftigung und die Sammlung von Erfahrungen zu zahlreichen, unsere bisherige Ausbildung ergänzenden 
Es war zudem sehr interessant zu erleben. - \textit{David Maier}
\par
\endgroup
