\subsection{Allgemein}            
Die Aufgabe der Hardwarekomponenten besteht darin, die in der Anwendung eingegebenen Befehle, bzw. die erstellten Zeichnungen real abzuarbeiten. Die jeweilige Verwendete Hardware soll alle von ihr unterstützten Befehle möglichst genau abarbeiten und dem Benutzer ein besseres Verständnis vermitteln welche Schritte und Bauteile nötig sind um einen Roboter zu betreiben.
Ein weiterer Aspekt dieser Komponente ist es für Präsentationsveranstaltungen der Schule besseres Anschauungsmaterial zu bieten.

Da bei der Entwicklung der Hardwareschnittstelle sehr viel Wert auf die Unabhängigkeit von der verwendeten Hardware gelegt wurde, war es möglich zwei unterschiedliche Hardwarekomponenten zu erstellen. Zum einen existiert ein funktionstüchtiger Roboter aus Holz undzum anderen wurde die Ansteuerung zweier Schrittmotoren der Firma KEBA realisiert.
Die Holzausfertigung des Roboters trägt in dieser schriftlichen Ausfertigung den Namen “Edubot Modell”. 