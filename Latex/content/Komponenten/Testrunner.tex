
\subsection{Testrunner}

\subsubsection{Aufgaben}
Die Aufgabe des Testrunner ist es die von uns erstellte Android Applikation mit Hilfe des Emulators zu testen.

\subsubsection{Umsetzung}
Der Testrunner ist ein JUnit 4 Testprojekt, welches mit Hilfe der \textit{MonkeyRunner} API von Google auf den Android Emulator zugreift. \\
Die \textit{MonkeyRunner} API ist eine Java Library. Sie ist Bestandteil des Android SDK und ermöglicht die Kommuniation mit der \textit{Android Debug Bridge} via Java. Mit der Library können Dateien auf das Gerät bzw. Emulator  kopiert werden und Eingabefehle wie ein Klick oder eine Texteingabe gesendet werden. \\[0.5em]
Da bevor ein Test gestartet werden kann gewisse Vorraussetzungen, wie zum Beispiel das starten des Emulators, getroffen werden müssen, wurde die Klasse \textit{MonkeyTest} entwickelt. Die \textit{MonkeyTest} erledigt alle Vorraussetzungen für einen Test in den \textit{setUp} und \textit{setUpBeforeClass} Methoden. Die Methode \textit{setUpBeforeClass} wird wie in JUnit üblich vor dem Ausführen einer Testklasse aufgerufen. \textit{setUp} hingegen wird vor jeder Testmethode aufgerufen. Eine neue Testklasse muss also nur von \textit{MonkeyTest} ableiten um funktionsfähig zu sein. \\
Weiters wurde eine Klasse \textit{MonkeyDevice} implementiert. Diese Klasse dient als Sicht zwischen dem Test und dem Emulator. \\
Folgende Aktionen werden von der \textit{setUpBeforeClass} Methode eines \textit{MonkeyTests} ausgeführt:
\begin{enumerate}
\item{\textbf{Laden des ADB}} \\
In diesem Schritt wird die \textit{MonkeyRunner} API initalisiert. Dazu muss der Pfad der zur lokalen Installation des Android SDK führt als VM-Argument mitgegeben werden.
\item{\textbf{Starten des Android Emulators}} \\
Wie auch der SDK Pfad muss auch der Emulatorenname als VM-Argument übergeben werden. In diesem Schritt wird dann geprüft ob dieser Emulator bereits läuft, wenn nicht wird er gestartet.
\item{\textbf{Installieren der Applikation}} \\
Hier wird über die ADB die APK-Datei installiert. Der Pfad der Datei muss als VM-Argument mitgeben werden.
\end{enumerate}
In der Methode \textit{setUp} werden weitere Vorraussetzungen für einen Test durchgeführt:
\begin{enumerate}
\item{\textbf{Kopieren der Testdaten auf den Emulator}} \\
Vor jedem Test werden speziell vorbereitete Datenbanken auf den Emulator kopiert. So kann trotz neuer Installation ein Login ohne Serververbindung durchgeführt werden.
\item{\textbf{Vorbereiten des Emulators}} \\
In diesem Schritt wird überprüft ob der Emulator gesperrt ist. Ist dies der Fall wird er entsperrt. Außerdem wird die lokale Zeiteinstellung des Emulators zurückgesetzt.
\item{\textbf{Beenden der Applikation}} \\
Läuft die Applikation auf dem Emulator bereits, wird sie beendet.
\item{\textbf{Ausführen von SQL-Statements}} \\
Im Konstruktor der \textit{MonkeyTest}-Klasse können optional SQL-Statements, wie das Einfügen eines Tiers, mitgeben werden. Diese Statements werden hier auf der vorher kopierten Datenbank ausgeführt.
\item{\textbf{Ausloggen aller Benutzer}} \\
Um Auch einen Start ohne Login simulieren zu können, kann im \textit{MonkeyTest}-Konstruktor dieses Flag mitgegeben werden. Dabei wird einfach das \textit{loggedIn} Attribut in der Benutzerdatenbank bei allen auf 0 also \textit{false} gesetzt.
\item{\textbf{Starten der Activity}} \\
Im \textit{MonkeyTest}-Konstruktor muss ein Activity-Pfad mitgegeben werden (zB.:\\\textit{"at.mkw.inlocs.android.ui.login.LoginActivity"}). Diese Actvity wird dann in diesem Schritt gestartet.
\end{enumerate}
\subparagraph{Schreiben eines Tests}
Mittels der Klasse \textit{MonkeyDevice} stehen dem Entwickler folgende Methoden zur Verfügung um auf den Emulator zuzugreifen:
\begin{itemize}
\item{\textit{touch(int x, int y)}} \\
Führt einen Klick bei den gegeben x und y Koordinaten aus.
\item{\textit{drag(int x1, int y1, int x2, int y2)}} \\
Führt eine Ziehbewegung von Punkt 1 zu Punkt 2 aus.
\item{\textit{sleep(double seconds)}} \\
Pausiert den Testrunner führ die übergebene Anzahl von Sekunden, dies ist öfters nötig da im Emulator das Öffnen von Dialogen oder Actvities etwas dauern kann.
\item{\textit{typeInEditText(int x, int y, String text, boolean clearBeforeType)}} \\
Setzt den Cursor in das Eingabefeld beim übergebenen Punkt und schreibt den mitgegebenen Text hinein. Wird bei \textit{clearBeforeType} \textit{true} übergeben, wird der Inhalt des Eingabefelds vor der Eingabe gelöscht.
\item{\textit{compareScreenshot(String imageName)}} \\
Nimmt einen Screenshot vom Emulator und vergleicht ihn mit einer Bilddatei am lokalen Filesystem. Dies wird verwwendet um zu erkennen ob ein Test erfolgreich durchgeführt wurde. Ist das Ergebnis nicht identisch wird ein Bild erstellt, welches das erwartete und das erhaltene nebeneinander stellt und am Dateisystem gespeichert. 
\end{itemize}
Ein Test würde also zum Beispiel so aussehen:
\begin{lstlisting}[language=java, captionpos=b, caption={Test Beispiel}]
   @Test
   public void testAddAnimal() throws Exception {
      device.touch(445,80);
      device.sleep(3);
      device.typeInEditText(40, 227, "Rudi", false);
      device.typeInEditText(40, 332, "1234", false);
      device.touch(240, 410);
      device.sleep(1);
      device.touch(122, 762);
      device.sleep(2);
      device.touch(122, 762);
      device.sleep(1.5);
      assertTrue(device.compareScreenshot("addAnimal"));
  }
\end{lstlisting}
\subparagraph{Integration mit Jenkins CI}
Für den Testrunner wurde auf den Jenkins Server der MKWe ein Job erstellt. Dieser Job wird nach dem Build der Android Applikation automatisch gestartet. Der Job führt ein Ant-Skript aus welches die gerade gebaute APK-Datei vom Server lädt und danach die Tests ausführt. Damit man die Ergebnisse bzw. Fehlschläge von der Jenkins-Weboberfläche betrachten kann wurde die Methode \textit{compareScreenshot} so bearbeitet, das sie den Vergleich des erwarteten und erhaltenen Screenshots auf den Server hochlädt. 