
\subsection{Hauptfenster}

\subsubsection{Aufgaben}
Das Hauptfenster ist das Hauptmenü des Programms und dient als zentrale Kreuzung im Programm. Durch das Hauptfenster ist es möglich zwischen den jeweiligen Punkten zu wechseln.

\subsubsection{Aufbau}
Das Hauptfenster besteht aus zwei Hauptbereichen. Beide Bereiche werden mittels ViewFlow Ansichten dargestellt. Diese Ansichten beinhalten Buttons, mit denen man sich durch die Applikation navigieren kann.
\begin{itemize}
\item{Die Ordner-Ansicht}\\
Diese Ansicht beinhaltet zwei Buttons, die zu den zwei großen Hauptbereichen Tierverwaltung und Gesundheitsverwaltung führen.

\item{Die Hinzufügen-Ansicht}\\
Diese Ansicht beinhaltet ebenfalls zwei Buttons, die direkt zu den jeweiligen Erstellungsansichten der Hauptbereichen führen. 
Der erste Button führt direkt in die Erstellungsansicht eines Tieres. Somit kann der Landwirt sehr schnell und einfach ohne Zwischensprünge ein Tier hinzufügen. 
Der zweite Button führt zur Behandlungsmaske. Hierbei gilt auch, dass der Landwirt schnell und einfach Verabreichungen und Auffälligkeiten zu einem Tier hinzufügen kann.
\end{itemize}
Beide dieser Ansichten besitzen eine Titelleiste, die einen Button für die Zentrale Suche beinhaltet. Ebenfalls besitzen beide Ansichten eine Leiste am Fuße des Bildschirms. Diese besteht aus zwei Ansichten, wobei die erste Ansicht den derzeitigen Status der Synchronisierung und die zweite Ansicht den derzeit angemeldeten Benutzer anzeigt.
\subsubsection{Umsetzung}
Wie auch in all den anderen Bereichen wird auch hier, eine ViewFlow eingesetzt. Sie ermöglicht es mehrere Ansichten zu verwenden und zwischen diesen zu wechseln. Auch hier wurde auf die Abstraction zurückgegriffen, die viele Funktionen bereits beinhaltet. Dieser Screen wird durch die HomescreenActivity dargestellt, diese Activity ist eine Ableitung der AbstractMultiscreenActivity. 

Das Grundlayout dieser Activity sieht folgendermaßen aus:
\begin{figure}[H]
\centering
\includegraphics[width=6cm]{images/app_screenshots/homescreen}
\caption{Die HomescreenActivity}
\end{figure} 

Wie man sieht besitzt diese Activiy eine Titelleiste, diese ist unterschiedlich zu den anderen Titelleisten, da sie nur den \includegraphics[width=0.7cm]{images/app_screenshots/search_button} -Button besitzt. Klickt man auf diesen Button, so wird man auf die zentrale Suche weitergeleitet. Diese zentrale Suche wurde von Herrn DI Pflug implementiert und ist daher nicht Bestandteil unserer Arbeit.

\subparagraph{Die StatusBar}

Am Fuße des Bildschirm ist die StatusBar zu sehen. Sie ist ein von uns entwickeltes UI-Element und besteht aus zwei Teilen.

\begin{itemize}

\item{Die SyncBar}\\
Diese SyncBar beinhaltet eine TextView, diese wird dazu verwendet, um den Status der Synchronisierung anzuzeigen. 

Es gibt 4 Möglichkeit wie der Status lautet:
\begin{itemize}
\item{Synchronisierung läuft}
\item{Synchronisierung erfolgreich}
\item{Synchronisierung fehlgeschlagen}
\item{Letzte Synchronisierung: + Zeitpunkt}
\end{itemize}
Weiters beinhaltet die SyncBar eine ImageView mit einem Synchronisierungssymbol als Inhalt. Bei laufender Synchronisierung wird eine Animation gestartet die den Button drehen lässt. Diese Animation wurde bereits in \textit{ Kapitel 3.2 Serveranmeldung} genau beschrieben. 
Mit dem \includegraphics[width=0.7cm]{images/app_screenshots/sync_bar_user_button} -Button kann man zur LoginBar wechseln, dieser Wechsel wurde durch eine TranslateAnimation gelöst. Dabei schiebt die Loginbar, von der rechte Seite ausgehend, die Syncbar am linken Rand hinaus.

Die implementierte Animation:

\begin{lstlisting}[language=java, captionpos=b, caption={Die Animation zum Wechsel der StatusBarViews}]
Animation animation = new TranslateAnimation(0, 
                                               loginBarWidth * (-1), 0, 0);

animation.setDuration(150);
animation.setFillEnabled(true);
animation.setFillBefore(true);
animation.setAnimationListener(new AnimationListener() {
   public void onAnimationStart(Animation animation) {
   }
   public void onAnimationRepeat(Animation animation) {
   }
   public void onAnimationEnd(Animation animation) {
      bottomBar.setLayoutParams(loginParams);
   }
});
\end{lstlisting}
 

\item{Die LoginBar}\\
Die LoginBar besteht ebenfalls aus einer TextView, jedoch zeigt diese den derzeit angemeldeten Benutzer. Der größte Unterschied zwischen den beiden Bars ist der, dass die gesamte Leiste klickbar ist. Klickt man sie, erscheint ein Kontextmenü das als AlertDialog implementiert wurde. In diesem Kontextmenü gibt es die Option \textit{"Abmelden"}. Wie der Name schon sagt, dient diese Option zum Abmelden und man wird auf die LoginActivity weitergeleitet.

\begin{figure}[H]
\centering
\includegraphics[width=6cm]{images/app_screenshots/logout}
\caption{Abmelden Kontextmenü}
\end{figure}

Die LoginBar besitzt ebenfalls einen Button zum Wechseln zwischen den beiden Bars. Jedoch besitzt dieser noch die Besonderheit, dass sich der SyncIcon des Buttons bei laufender Synchronisierung dreht. Diese Dreh-Animation wurde bereits erklärt. Durch das Klicken des Buttons wird die selbe Aktion wie bei dem vorherigen Button in Gang gesetzt, jedoch geht jetzt die Schiebe-Animation zur SyncBar und deshalb genau in die gegengesetzte Richtung.

\subparagraph{Die Ordner Ansicht}
Das Layout für die Ordner Ansicht wurde als TableLayout definiert. Mit diesem Layout ist es einfach möglich die Buttons in einem Raster anzuordnen. Die Buttons selber besteht aus einer Imageview, die das Icon des Buttons darstellen soll und einer Textview, die den Titel des Buttons anzeigen soll.

\begin{figure}[H]
\centering
\includegraphics[width=6cm]{images/app_screenshots/homescreen_ordner}
\caption{Die Ordner Ansicht}
\end{figure}

Durch das Klicken des jeweiligen Buttons, wird man entweder zur Tierverwaltung oder zur Gesundheitsverwaltung weitergeleitet.



\subparagraph{Die Hinzufügen Ansicht}
Diese Ansicht besitzt das selbe TableLayout wie die Ordner Ansicht und unterscheidet sich nur in der Aufgabe der Buttons. 
Durch das Klicken ihrer Buttons kommt man nämlich direkt in die Erstellungsansichten. Wie schon erwähnt wird somit das Erstellen von Objekten wesentlich beschleunigt, da sich der Landwirt mit einem Schritt in den verschiedenen Erstellungsansichten wieder findet.

\begin{figure}[H]
\centering
\includegraphics[width=6cm]{images/app_screenshots/homescreen_adding}
\caption{Die Hinzufügen Ansicht}
\end{figure}

\end{itemize}






