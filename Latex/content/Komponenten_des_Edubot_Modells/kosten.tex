\subsection{Kostenaufstellung}
Die Kosten für den Bau des Edubot Modells wurden zum großteil von der HTL Grieskirchen übernommen, da das fertige Modell mit Projektabschluss auch der Schule übergeben wurde. Beim Bau des Roboters entstanden folgende Kosten:

\begin{tabular}{|p{11cm}|p{3cm}|}
\hline \rowcolor{lightgray}
\textbf{Verwendungszweck} & \textbf{Kosten}\\
\hline
2 Schrittmotoren der Firma Nanotech & [todo]\\
\hline
2 Schrittmotorsteuerungen der Firma Nanotech & [todo]\\
\hline
1 Laser & 55\euro{}\\
\hline
4 Kugellager & 19\euro{}\\
\hline
1 Microcontroller vom Typ GHI Embedded Master Breakout Board 1.0 (gebraucht) & 30\euro{}\\ 
\hline
2qm Dreischichtplatte & 30\euro{}\\ 
\hline
Schrauben und sonstige Kleinteile (gebraucht) & 30\euro{}\\ 
\hline
\end{tabular}