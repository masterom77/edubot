\subsection{Allgemein}            
Der Name Edubot bezeichnet sowohl das Softwaresystem als ganzes, als auch den aus Holz gefertigten Roboterarm. Der Roboterarm wird allgemein als "'Edubot Modell"' bezeichnet.

\subsubsection{Funktionen}
Das Edubot Vorführmodell verfügt über zwei rotatorische Achsen die auf horizontaler Ebene bewegt werden können. 
Der Roboter besitzt damit zwei Freiheitsgrade und kann auf einer Ebene alle Punkte in einem Nieren förmigen Arbeitsbereich anfahren. Abhängig vom montierten Werkzeug kann der Roboter beispielsweise zeichnen oder mit Hilfe eines geeigneten Lasers dünne Materialien wie Moosgummi schneiden.

\subsubsection{Bedienung}
Als Steuerungsteil des Roboterarms dient ein Mikrocontroller (GHI Embedded Master) welcher sowohl über eine Netzwerkschnittstelle, als auch über einen USB Anschluss verfügt. Zur Übergabe von Befehlen an den Roboter wird ausschließlich die Netzwerkschnittstelle verwendet, da der USB Anschluss nur benötigt wird um Softwareänderungen am Controller durchzuführen.
Um dem Roboterarm Befehle zu übergeben muss der Mikrocontroller mithilfe eines normalen Netzwerkkabels mit RJ-45 Steckern an das Netz angeschlossen werden in welchem sich ein Computer mit einsatzfähiger Edubot Software befindet (das Netzwerkkabel kann auch direkt an einen Computer angeschlossen werden).