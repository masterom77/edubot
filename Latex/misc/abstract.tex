
\phantomsection
\addcontentsline{toc}{section}{Abstract}

\section*{Abstract}
Edubot is a simple-to-use, functional and effective tool for learning the operating principles of a robot. It will be used in the subject "process control with laboratory exercises" during the “robotics” part of the third schoolyear. The project includes the construction of a simple robot with two gyratory axes, an application programming interface (API) which enables the students to quickly and easily integrate the robot into their own programs, as well as an end-use application that can be used for presentation and teaching purposes. 

\section*{Kurzfassung}
Diese Diplomarbeit ist ein Bestandteil des Projektes \textit{SmartBow} des Unternehmens \textit{MKW electronics}. Die Aufgabe von \textit{SmartBow} ist es die täglich anfallenden Tätigkeiten in einem Landwirtschaftsbetrieb zu vereinfachen. Besonders wert gelegt wird dabei auf die Verwaltung von Tieren und deren Symptomen und Medikationen. Von den Mitarbeitern der Firma \textit{MKWe} wurde schon im Vorfeld ein Desktop-Client, mit den entprechenden Funktionen, sowie ein zentraler Server, der die Daten verwaltet, programmiert. Die konkrete Aufgabe dieser Arbeit war es eine Android Applikation zu erstellen, die dieselben Funktionen besitzt wie der Desktop-Client. Aufgrund des agilen Softwareentwicklungsvorganges des Unternehmens war enge Zusammenarbeit vorrausgesetzt und einfach erweiterbarer und verständlicher Code ein Hauptziel.


%
% EoF
%